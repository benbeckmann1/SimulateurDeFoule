\documentclass{article}
\usepackage[T1]{fontenc}
\usepackage{a4wide}
\usepackage{graphicx}
\usepackage{amssymb}
\usepackage{amsmath}
\usepackage{color}
\usepackage{babel}
\usepackage[utf8]{inputenc}   
\usepackage[T1]{fontenc}
\usepackage{newunicodechar}
\usepackage{subcaption}
\usepackage[export]{adjustbox}
\usepackage{wrapfig}
\begin{document}


\begin{figure}[t]
\centering
\includegraphics[width=10cm]{n7.png}
\end{figure}

\title{\vspace{4cm} \textbf{Liste des fonctionnalités du projet de technologie objet\\Simulateur de foule}}
\author{de Brandois Félix, Cognard Clément, El Guerraoui Oussama, Fresco Alan\\ Mimoun Ibtissam, Fraine Sofiane, Vithursan}
\date{\vspace{7cm} Département Sciences du Numérique - Premiére année - 2022-2023 }

\maketitle

\newpage
\tableofcontents
\listoffigures

\newpage
\section{Objectif général du projet}
Ce projet est un simulateur qui représente de manière précise le déplacement de personnes dans un bâtiment. Il pourra prendre en compte plusieurs situations type (situation d’urgence, bâtiment surchargé de monde…). Il sera capable d’évaluer le temps d’évacuation de la foule. Il pourra être programmable, c’est-à-dire que l’on pourra choisir la disposition des murs, des obstacles, de la sortie du lieu.

\section{Description des fonctionnalités}
    \subsection{Programmable}
    L'utilisateur pourra configurer le batiment simulé, il sera en mesure de choisir l'emplacement des murs, de la sortie et du nombre de personnes.
    
    \subsection{Personnes configurables}
    Il sera possible de définir les caractéristiques des personnes dans le simulateur afin de se rapprocher d'une situation réelle (taille, poids, vitesse, résistance, ...).

    \subsection{Phénomènes extérieurs}
    Le logiciel prendra en compte la gestion de phénomènes extérieurs comme un incendie, un bâtiment surchargé de monde, un attentat, ...
        
    \subsection{Panique}
    La panique interviendra sur le comportement de l'individu lors de l'évacuation : modification de la vitesse. 

    \subsection{Ajout d'obstacles}
    L'utilisateur pourra placer lui-même des obstacles et choisir leur caractéristiques (mobiles : chaises, statiques : piliers).

    \subsection{Affichage de la pression physique}
    Le logiciel devra afficher la pression physique subie par les personnes via des couleurs plus ou moins intenses.
    
    \subsection{Rapports automatiques}
    A la fin de la simulation, différentes courbes seront affichées, ces dernières montreront plusieurs statistiques relatives à l'évacuation telles que le temps de sortie moyen, la pression maximale subie, le nombre de morts, ... 

    \subsection{Scan de plan}
    Le logiciel doit être capable de scanner des plans et de lancer une simulation directement dessus.
    
\section{Interface utilisateur envisagée}
     Voici les interfaces envisagées :
    \begin{figure}[ht!]
        \begin{subfigure}{0.5\textwidth}
            \includegraphics[width=0.9\linewidth, height=6cm]{Diagram - CrowdSimulator (3).png}
            \caption{Page d'accueil du logiciel \label{fig : PageAccueil}}
        \end{subfigure}
        \begin{subfigure}{0.5\textwidth}
            \includegraphics[width=0.9\linewidth, height=6cm]{Diagram - CrowdSimulator.png}
            \caption{Page d'une simulation de base \label{fig : SimulationBase}}
        \end{subfigure}\\
        \begin{subfigure}{0.5\textwidth}
            \includegraphics[width=0.9\linewidth, height=6cm]{Diagram - CrowdSimulator (2).png}
            \caption{Page d'une simulation aléatoire \label{fig : SimulationAléatoire}}
        \end{subfigure}
        \begin{subfigure}{0.5\textwidth}
            \includegraphics[width=0.9\linewidth, height=6cm]{Diagram - CrowdSimulator (1).png}
            \caption{Page d'une simulation personalisée \label{fig : SimulationPerso}}
        \end{subfigure}
        \caption{Interface utilisateur \label{fig : Interface}}
    \end{figure}

\section{Scénarios}
    \subsection{Incendie}
    Le logiciel doit être capable de simuler l'évacuation d'une foule lors d'un incendie.
    
    \subsection{Stade}
    Il doit être possible de simuler l'évacuation d'une foule lors de grands évènements.
    
    \subsection{Attentats}
    Le logiciel doit être capable de simuler l'évacuation d'une foule lors d'un attentat.
    
    \subsection{Ecole}
    Le logiciel doit être capable de simuler le déplacement des élèves lors d'un changement de salle.
    
\section{Difficultés}
    \begin{itemize}
        \item Gestion des collisions
        \item Modélisation du comportement humain
        \item Implémentation de l'IHM
        \item Scan de plan
        \item Implémentation en 3D
    \end{itemize}

\end{document}
